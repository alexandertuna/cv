% packages
\documentclass{resume2}
\usepackage{hyperref}
\usepackage{color}
\usepackage{enumitem}
\usepackage[left=0.8in,top=0.8in,right=1.4in,bottom=0.8in,nofoot]{geometry}

\usepackage{amsmath}
\usepackage{slashed}

\pagestyle{empty}

% begin
\begin{document}

\name{\huge{Alexander Tuna}}
\address{\Large{curriculum vitae}}

\begin{resume}

% contact
\sectionX{Contact information}
\vspace{0.2cm}

\hspace{-0.2cm}\begin{tabular}{l l}
E-mail & \underline{\color{blue}{\url{tuna@cern.ch}}} \\
Phone (US) & (336) 749-8522 \\
Phone (CH) & +41 076 487 9012 \\
Address & 209 S. 33rd St. \\
        & David Rittenhouse Laboratories \\
        & Philadelphia, PA 19104 \\
\end{tabular}

% education section
\sectionX{Education}
\vspace{0.2cm}

% penn
\textbf{University of Pennsylvania}, Philadelphia, Pennsylvania. 2010 - present. \\
Ph.D. in Physics, April 2015 \\
Dissertation topic: \textit{Evidence for $H\!\rightarrow\!\tau\tau$ at ATLAS} \\
Advisor: Prof. Hugh ``Brig'' Williams \\
M.S. in Physics, May 2014

% duke
\textbf{Duke University}, Durham, North Carolina. 2006 - 2010. \\
B.S. in Physics, minor in Mathematics, May 2010 \\
Senior thesis: \textit{Search for Fractionally-Charged Particles at Super-Kamiokande} \\
Advisor: Prof. Chris Walter \\
Daphne Chang Memorial Award for outstanding undergraduate research

% research section
\sectionX{Research}
\vspace{0.2cm}

% atlas
\textbf{ATLAS Experiment} at the LHC, Geneva, Switzerland. 2010 - present.
\begin{itemize}

\item Ph.D. student for the Penn ATLAS Group
\item Evidence for $H\!\rightarrow\!\tau\tau$ (2013 - present) [1, 3, 5]
  \begin{itemize}
  \item I led the development of the data-driven prediction of the dominant background ($j\!\rightarrow\!\tau_h$ fakes) for the $H\!\rightarrow\!\tau_\ell\tau_h$ channel. The prediction was made more data-driven by including $Z+j$ and $t\overline{t}$ processes, and the systematic uncertainties were reduced from 20-40\% to 5\% with comprehensive tests of the model and by tightening the extrapolation region.
  \item I revamped the $\tau_h$ identification criteria, which led to 10\% improvement in signal acceptance and increased rejection of $\mu\!\rightarrow\!\tau_h$ fakes.
  \item I am a core developer of the analysis software for the $H\!\rightarrow\!\tau_\ell\tau_h$ channel, which is used for all official results since summer 2013. 
  \item I am the liaison to the harmonization group and trigger group, and have helped coordinate discussion with the electron performance group, muon performance group, $\slashed{E}_{\text{T}}$ performance group, and derivation framework developers.
  \item A preliminary version of this analysis was presented at a CERN seminar in autumn 2013 [3] and reported first ever evidence ($4.1\sigma$) of $H\!\rightarrow\!\tau\tau$ decays. A publication describing the final analysis is currently in progress [1].
  \end{itemize}

\item Tau triggers for Run-II (2013 - present)
  \begin{itemize}
  \item I am leading the development of the tau trigger menu for Run-II.
  \item I investigated using topological variables at L1 as an early adopter of \texttt{L1topo}. I found $\Delta R(\tau, \tau)$ can be used to significantly reduce the L1 rate for di-$\tau_\text{h}$ triggers with negligible physics loss.
  \item I developed the new $p_\text{T}$-dependent L1 isolation menu which recovers significant efficiency loss at high $p_\text{T}^\tau$.
  \item These will be the primary triggers in 2015 for the $H\!\rightarrow\!\tau_\text{h}\tau_\text{h}$ and $HH\!\rightarrow\!bb\tau_\text{h}\tau_\text{h}$ analyses, among others.
  \end{itemize}

\item TRT DAQ (2014 - present)
  \begin{itemize}
  \item I am part of the TRT DAQ team preparing for data-taking in 2015 and beyond, when the L1 accept rate will increase from 75 kHz to 100 kHz and the TRT occupancy will increase according to pile-up.
  \item I am helping monitor the performance of off-detector electronics and on-detector power supplies, and to replace them as necessary.
  \item The TRT has successfully recorded cosmic data events in recent ``milestone weeks'', and the TRT Fast-OR trigger has provided cosmic data events to the rest of ATLAS.
  \end{itemize}

\item Prospects for $H\!\rightarrow\!\tau\tau$ at the HL-LHC (2014 - present) [8]
  \begin{itemize}
  \item I am responsible for investigating prospects for measuring VBF $H\!\rightarrow\!\tau\tau$ at HL-LHC conditions: 14 TeV, $3000\text{ fb}^{-1}$, and $\mu = 140$.
  \item This is part of a broader effort to evaluate the physics case of a forward tracker ($|\eta| > 2.5$) being considered for Phase-II upgrades.
  \item The impact of forward pile-up suppression had significant impact on the analysis -- the uncertainty on the measurement ($\Delta\mu$) was improved from 24\% to 8\% with pile-up suppression of VBF jets via forward tracking, providing general motivation for forward tracking for VBF analyses.
  \item This study was presented at the European Committee for Future Accelerators (ECFA) High Luminosity LHC Experiments Workshop in late 2014 [8].
  \end{itemize}

\item Tau performance: $e\!\rightarrow\!\tau_\text{h}$ discriminant development and efficiency measurement (2012 - 2013) [2, 4]
  \begin{itemize}
  \item I measured the efficiency of the preliminary $e\!\rightarrow\!\tau_\text{h}$ discriminant (``electron veto'') in data and simulation using $Z \!\rightarrow\! ee$ $(e\!\rightarrow\!\tau_\text{h})$ tag-and-probe. This BDT-based discriminant is commonly used in analyses with $\tau_\text{h}$ since electrons typically pass the $j\rightarrow\tau_\text{h}$ discriminant. 
  \item I uncovered a mis-modeling of the energy leakage in the third layer of the EM calorimeter for $e\!\rightarrow\!\tau_\text{h}$ fakes with $|\eta| > 2$, and this propagated to a difference in the performance of the discriminant in data versus simulation of $\sim\!10\times$.
  \item To ameliorate this mis-modeling, I derived a well-modeled replacement for the offending variable and re-trained the $e\!\rightarrow\!\tau_\text{h}$ discriminant. I additionally pruned the variable list for redundancy and retrieved high-statistics samples for training, such that the updated discriminant was better modeled and more performant with fewer input variables.
  \item I measured the efficiency of the updated $e\!\rightarrow\!\tau_\text{h}$ discriminant in data and simulation with the same methods. I also observed $e\!\rightarrow\!\tau_{h,3p}$ fakes from ``trident'' electrons for the first time, and I measured the efficiency of these to pass the discriminant in data and simulation.
  \item This work was presented in a preliminary description of tau performance at ATLAS in summer 2013 [4]. A publication describing tau performance is currently in progress [2].
  \end{itemize}

\item Search for $Z' \!\rightarrow\! \tau\tau$ (2011 - 2012) [6, 7]
  \begin{itemize}
  \item I developed and performed a search for a new heavy neutral boson ($Z'$) decaying to $\tau_{\ell}\tau_\text{h}$ in collaboration with Ryan Reece. This was the first search for heavy resonances decaying to tau(s) at ATLAS.
  \item We developed novel background estimation techniques for $j\!\rightarrow\!\tau_\text{h}$ fakes (``fakefactors'') which are now commonly used among analyses with $\tau_\text{h}$ in the final state.
  \item I uncovered issues with tau performance at high $p_{\text{T}}$, including a non-optimal tuning of the $e\!\rightarrow\!\tau_\text{h}$ fakes discriminant, which I re-tuned, and the issue of mis-classification of 3-prong $\tau_\text{h}$ as 2-track due to the merging of closeby tracks, which is an unresolved issue for high $p_{\text{T}}$ tracking.
  \item A preliminary version of this analysis was presented at ICHEP in summer 2012 [7] in combination with the $\tau_\text{h}\tau_\text{h}$ and $\tau_{\ell}\tau_{\ell}$ channels. The final analysis was published in Phys. Lett. B in late 2012 [6] and set the strongest limit on SSM $Z'\!\rightarrow\!\tau\tau$ decays at the time of publication.
  \end{itemize}

\end{itemize}

\pagebreak

% super-k
\textbf{Super-Kamiokande}, Kamioka, Japan. 2007 - 2010.
\begin{itemize}
\item I worked as an undergraduate researcher for Prof. Chris Walter of the Duke Neutrino Group for three years (excluding summer 2009).
\item I developed and performed a search for fractionally-charged particles (FCP) in the cosmic rays at the Super-Kamiokande water Cherenkov detector.  
  \begin{itemize}
  \item I modified Monte Carlo simulation programs and particle reconstruction algorithms to accurately simulate FCP at Super-K.
  \item The search over a partial SK-II dataset placed a limit on the flux of FCP which was competitive with previous searches.
  \end{itemize}
\item I helped build cables for SK-IV electronics upgrade.
\end{itemize}

\sectionX{}

% super-k
\textbf{CERN Technology Department}, Geneva, Switzerland. 2009.
\begin{itemize}
\item I worked as a summer student for Dr. Michael Koratzinos in the CERN TE-MPE group as part of the Michigan CERN REU for 3 months.
\item I helped identify potential problems in the super-conducting magnets of the LHC by exploring the temperature dependence of the resistance of the copper busbars.
\end{itemize}

% other shit
\sectionX{Work}
\vspace{0.2cm}

% TA
\textbf{University of Pennsylvania}, Philadelphia, Pennsylvania \\
Teaching Assistant, Department of Physics and Astronomy (four semesters, 2010-2012) \\
I taught labs and graded homework, exams, and labs for four semesters of introductory undergraduate courses in Newtonian mechanics and electromagnetism.

\sectionX{}

% mather
\textbf{AIP Mather Policy Internship Program}, Washington, DC \\
Intern, Office of Congressman Bill Foster D-IL14 (summer 2010) \\
I helped the legistlative staff with constituent correspondence and services. I also attended hearings in Congress regarding science policy and provided summaries to the staff.

% public shit
\sectionX{Presentations}
\vspace{0.6cm}
\begin{itemize}[leftmargin=*]
  % \item \textit{Higgs bosons and tau leptons at ATLAS}. University of Chicago HEP Lunch Seminars. Chicago, June 2015.
  \item \textit{Higgs bosons and tau leptons at ATLAS}. University of Pennsylvania Experimental Particle Physics Seminars. Philadelphia, April 2015.
  \item \textit{Evidence for $H\rightarrow\tau\tau$ at ATLAS}. 50th Rencontres de Moriond (EW). La Thuile, March 2015.
  \item \textit{Evidence for $H\rightarrow\tau\tau$ at ATLAS}. University of Pittsburgh PITT PACC Seminar. Pittsburgh, January 2015.
  \item \textit{Prospects for Higgs searches in the ditau channel in Run 2 at ATLAS}. US ATLAS Physics Workshop 2014. Seattle, Washington, August 2014. {\color{red}{ATLAS Internal}}.
  \item \textit{Searches for decays of the Higgs-like boson to tau lepton pairs with the ATLAS detector}. Meeting of the American Physical Society Division of Particles and Fields. Santa Cruz, California, August 2013.
  \item \textit{Searches for decays of a Higgs boson to tau lepton pairs with the ATLAS detector} (poster). 26th International Symposium on Lepton Photon Interactions at High Energies. San Francisco, California, June 2013.
  \item \textit{Performance of Tau Reconstruction and Identification in 2012 with ATLAS} (poster). 113th LHCC Meeting. Geneva, Switzerland, March 2013.
\end{itemize}

\pagebreak

% references
\sectionX{References}
\vspace{0.5cm}
\noindent\begin{tabular}{l l}
Prof. Hugh ``Brig'' Williams & Mary Amanda Wood Professor, Department of Physics, University of Pennsylvania \\
                             & \underline{\color{blue}{\url{williams@physics.upenn.edu}}} \\
                             & (215) 898-6284 \\
& \\
Prof. Elliot Lipeles & Associate Professor, Department of Physics, University of Pennsylvania \\
                     & \underline{\color{blue}{\url{lipeles@hep.upenn.edu}}} \\
                     & (215) 573-3652 \\
& \\
Prof. Sinead Farrington & Associate Professor, Department of Physics, University of Warwick \\
                        & Co-Convener, ATLAS $H\!\rightarrow\!\tau\tau$ analysis group (2013 - present) \\
                        & \underline{\color{blue}{\url{sinead.farrington@cern.ch}}} \\
                        & +44 (0)2476 528044 \\
& \\
Prof. Attilio Andreazza & Professor, Dipartimento di Fisica, Universit\`{a} degli studi di Milano \\
                        & Co-Convener, ATLAS Tau Combined Performance group (2013 - present) \\
                        & \underline{\color{blue}{\url{attilio.andreazza@unimi.it}}} \\
                        & +39 02503 17375 \\
& \\
\end{tabular}

% other shit
\sectionX{Selected publications}
\vspace{-0.8cm}
\renewcommand{\refname}{\selectfont\normalsize}
\begin{thebibliography}{9}
\bibitem{htautaupaper}
  ATLAS Collaboration.
  \textit{Evidence for the Higgs-boson Yukawa coupling to tau leptons with the ATLAS detector}. 
  \href{http://atlas.web.cern.ch/Atlas/GROUPS/PHYSICS/PAPERS/HIGG-2013-32/}{\underline{\color{blue}{\texttt{HIGG-2013-32}}}}. 
  JHEP 04 (2015) 117.
  April 2015.
\bibitem{taupaper}
  ATLAS Collaboration.
  \textit{Identification and energy calibration of hadronically decaying tau leptons with the ATLAS experiment in $pp$ collisions at $\sqrt{s}$ = 8 TeV}. 
  \href{http://atlas.web.cern.ch/Atlas/GROUPS/PHYSICS/PAPERS/PERF-2013-06/}{\underline{\color{blue}{\texttt{PERF-2013-06}}}}.
  Submitted to EPJC.
  December 2014.
\bibitem{htautauLHCC}
  ATLAS Collaboration.
  \textit{Evidence for Higgs Boson Decays to the $\tau^{+}\tau^{-}$ Final State with the ATLAS Detector}. 
  \href{https://atlas.web.cern.ch/Atlas/GROUPS/PHYSICS/CONFNOTES/ATLAS-CONF-2013-108/}{\underline{\color{blue}{\texttt{ATLAS-CONF-2013-108}}}}. 
  November 2013. 
\bibitem{tauconf}
  ATLAS Collaboration.
  \textit{Identification of Hadronic Decays of Tau Leptons in 2012 Data with the ATLAS Detector}. 
  \href{https://atlas.web.cern.ch/Atlas/GROUPS/PHYSICS/CONFNOTES/ATLAS-CONF-2013-064/}{\underline{\color{blue}{\texttt{ATLAS-CONF-2013-064}}}}. 
  July 2013. 
\bibitem{htautauHCP}
  ATLAS Collaboration.
  \textit{Search for the Standard Model Higgs boson in $H\!\rightarrow\!\tau^{+}\tau^{-}$ decays in proton-proton collisions with the ATLAS detector}. 
  \href{https://atlas.web.cern.ch/Atlas/GROUPS/PHYSICS/CONFNOTES/ATLAS-CONF-2012-160/}{\underline{\color{blue}{\texttt{ATLAS-CONF-2012-160}}}}. 
  November 2012. 
\bibitem{zprimepaper}
  ATLAS Collaboration.
  \textit{A search for high-mass resonances decaying to $\tau^{+}\tau^{-}$ in pp collisions at $\sqrt{s}$ = 7 TeV with the ATLAS detector}. 
  \href{https://atlas.web.cern.ch/Atlas/GROUPS/PHYSICS/PAPERS/EXOT-2012-03/}{\underline{\color{blue}{\texttt{EXOT-2012-03}}}}. 
  Phys. Lett. B 719 (2013) 242-260. 
  October 2012.
\bibitem{zprimeconf}
  ATLAS Collaboration.
  \textit{A search for high mass resonances decaying to $\tau^{+}\tau^{-}$ in the ATLAS detector}. 
  \href{https://atlas.web.cern.ch/Atlas/GROUPS/PHYSICS/CONFNOTES/ATLAS-CONF-2012-067/}{\underline{\color{blue}{\texttt{ATLAS-CONF-2012-067}}}}. 
  June 2012.
\bibitem{tautauhllhc}
  ATLAS Collaboration.
  \textit{Studies of the VBF $\ensuremath{H\rightarrow\tau_\ell\tau_\text{h}}$ analysis at High Luminosity LHC Conditions}.
  \href{https://atlas.web.cern.ch/Atlas/GROUPS/PHYSICS/PUBNOTES/ATL-PHYS-PUB-2014-018/}{\underline{\color{blue}{\texttt{ATL-PHYS-PUB-2014-018}}}}. 
  October 2014.
\end{thebibliography}

%\sectionX{Publications}
%\vspace{0.6cm}
%\begin{itemize}[leftmargin=*]
%  \item \textit{Evidence for Higgs-Boson Yukawa couplings in the $H\!\rightarrow\!\tau^{+}\tau^{-}$ decay mode with the ATLAS detector} (in progress). \href{https://cds.cern.ch/record/1693466}{\underline{\color{blue}{ATLAS-HIGG-2013-32}}}. {\color{red}{ATLAS Internal}}. 
%  \item \textit{Identification and Energy Calibration of Hadronically-decaying Tau Leptons with the ATLAS Experiment in $pp$ Collisions at $\sqrt{s}$ = 8 TeV} (in progress). \href{https://cds.cern.ch/record/1747796}{\underline{\color{blue}{ATLAS-PERF-2013-06}}}. {\color{red}{ATLAS Internal}}.
%  \item \textit{Evidence for Higgs Boson Decays to the $\tau^{+}\tau^{-}$ Final State with the ATLAS Detector}. \href{https://atlas.web.cern.ch/Atlas/GROUPS/PHYSICS/CONFNOTES/ATLAS-CONF-2013-108/}{\underline{\color{blue}{ATLAS-CONF-2013-108}}}. November 2013. 
%  \item \textit{Identification of Hadronic Decays of Tau Leptons in 2012 Data with the ATLAS Detector}. \href{https://atlas.web.cern.ch/Atlas/GROUPS/PHYSICS/CONFNOTES/ATLAS-CONF-2013-064/}{\underline{\color{blue}{ATLAS-CONF-2013-064}}}. July 2013. 
%  \item \textit{Search for the Standard Model Higgs boson in $H\!\rightarrow\!\tau^{+}\tau^{-}$ decays in proton-proton collisions with the ATLAS detector}. \href{https://atlas.web.cern.ch/Atlas/GROUPS/PHYSICS/CONFNOTES/ATLAS-CONF-2012-160/}{\underline{\color{blue}{ATLAS-CONF-2012-160}}}. November 2012. 
%  \item \textit{A search for high-mass resonances decaying to $\tau^{+}\tau^{-}$ in pp collisions at $\sqrt{s}$ = 7 TeV with the ATLAS detector}. \href{https://atlas.web.cern.ch/Atlas/GROUPS/PHYSICS/PAPERS/EXOT-2012-03/}{\underline{\color{blue}{EXOT-2012-03}}}. Phys. Lett. B 719 (2013) 242-260, arXiv:1210.6604 [hep-ex]. October 2012.
%  \item \textit{A search for high mass resonances decaying to $\tau^{+}\tau^{-}$ in the ATLAS detector}. \href{https://atlas.web.cern.ch/Atlas/GROUPS/PHYSICS/CONFNOTES/ATLAS-CONF-2012-067/}{\underline{\color{blue}{ATLAS-CONF-2012-067}}}. June 2012.
%\end{itemize}

% end 
\end{resume}
\end{document}


